\documentclass[a4paper,11pt]{jreport}

\usepackage{times}
\usepackage{amsmath}
\usepackage{amsfonts}
\usepackage{amssymb}
\usepackage{braket}
\usepackage{amsthm}

\newtheorem{definition}{定義}
\newtheorem{theorem}{定理}
\renewcommand{\proofname}{証明}

\setcounter{tocdepth}{3}
\setcounter{page}{-1}

\setlength{\oddsidemargin}{0.1in}
\setlength{\evensidemargin}{0.1in} 
\setlength{\topmargin}{0in}
\setlength{\textwidth}{6in} 
\setlength{\parskip}{0em}
\setlength{\topsep}{0em}

\usepackage{sie-jp}

\title{単体上における非凸関数の大域的最適化}
\author{千葉 竜介}
\degree{修士(工学)}
\advisor{久野 誉人}

\majorfield{コンピュータサイエンス}
\yearandmonth{2017年 3月}

\begin{document}
\maketitle
\thispagestyle{empty}
\newpage

\thispagestyle{empty}
\vspace*{20pt plus 1fil}
\parindent=1zw
\noindent
\begin{center}
{\bf 概要}
\vspace{5mm}
\end{center}

% TODO: 応用例,過去の研究について述べる
凸関数最適化問題と比較して非凸関数の最適化は一般的に困難であると言われている.その中でも,非凸関数の一種である単調関数(Monotonic function)の最適化を扱う.単調関数とは,入力の増加に対して関数値も常に増加または減少する関数のことである.そのような単調関数について,実行可能領域を単体(Simplex)とした場合の最適化を考える.\par
本稿では,大域的最適化の手法としてよく利用されている分枝限定法を利用する.分枝限定法は実行可能領域の分割を繰り返し,たくさんの小さな問題を解く手法である.実行可能領域における関数値の下界と暫定解と比較することで,その領域に最適解が存在しうるかどうかを判断し,最適解が存在し得ない場合はその領域を削除して効率的に最適化を行う.したがって分枝限定法において,良い下界を与えること,効果的に実行可能領域を分割することは重要である.\par
単体の新たな分割方法の提案を行い,その分割手法が生成する子領域の数の上限を示す.さらに,提案した分割手法を使った分枝限定法の実験を行い,提案手法を評価する.

\par
\vspace{0pt plus 1fil}
\newpage

\pagenumbering{roman}
\tableofcontents
% \listoffigures
% \listoftables

\pagebreak
\setcounter{page}{1}
\pagenumbering{arabic}


\chapter{序論}
\section{概要}

% TODO: 応用例,過去の研究について述べる

\section{記号の定義}

\begin{table}[htb]
\begin{tabular}{ll}
表記 & 意味 \\ \hline
$ \mathbb{R} $ & 実数全体の集合 \\
$ \mathbb{R}^n $ & $ n $ 次元実数ベクトル全体の集合 \\
$ \mathbb{R}^n_{+} $ & すべての要素が非負である $ n $ 次元実数ベクトル全体の集合 \\
$ \mathbb{Z} $ & $ 0 $ を含む自然数全体の集合 \\
$ \mathbb{Z}^n $ & $ n $ 次元自然数ベクトル全体の集合 \\
$ e_i \; (e_i \in \mathbb{R}^n) $ & $ i $ 番目の要素が $ 1 $ でそれ以外が $ 0 $ であるような単位ベクトル \\
$ x_i \; (x \in \mathbb{R}^n) $ & ベクトル $ x $ の $ i $ 列目の要素 \\
$ x < y \; (x, y \in \mathbb{R}^n) $ & $ x_i < y_i \; (i = 1, ..., n) $ \\
$ x \leq y \; (x, y \in \mathbb{R}^n) $ & $ x_i \leq y_i \; (i = 1, ..., n) $
\end{tabular}
\end{table}


\section{構成}

\chapter{単体上の単調関数最適化問題}

本研究で扱う問題は以下のような単調関数 $ f $ の最小化問題である.

$$
\left| \;
\begin{aligned} \label{math:primal_problem}
& 最小化 && f(x) &&&&& \\
& 条件 && \sum_{i=1}^n x_i = 1 \notag \\
& && x_i \geq 0, i = 1, ..., n \notag
\end{aligned}
\right.
$$

制約条件に,$ x $ の各要素の和が $ 1 $ であることと非負であることがある.このような条件を満たす $ x $ の集合は単体と呼ばれており,様々な研究がなされている.この問題は単調関数の単体上における最適化問題であり,まずは単調関数と単体を定義し,それらの性質について論ずる.

\section{単調関数}

単調関数とは,単調増加関数または単調減少関数のことであり,入力の増加に対して関数値も常に増加または減少する関数である.\\

\begin{definition}
$ I \subseteq \mathbb{R}^n $ の関数 $ f : I \to \mathbb{R} $ は任意の $ x, x' \in I $ に対して以下が成り立つ場合に単調増加関数であるという.
$$ x < x' \Rightarrow f(x) \leq f(x') $$
\end{definition}

\begin{definition}
$ I \subseteq \mathbb{R}^n $ の関数 $ f : I \to \mathbb{R} $ は任意の $ x, x' \in I $ に対して以下が成り立つ場合に単調減少関数であるという.
$$ x < x' \Rightarrow f(x) \geq f(x') $$
\end{definition}

ここで $ I $ は $ n $ 次元実数空間全体や区間,離散的な空間が考えられる.\\

\begin{theorem} \label{keeping_monotonicity}
$ f : I \to \mathbb{R} $ が単調増加関数であるとき,$ -f $ は単調減少関数である.逆も同様に成り立つ.
\end{theorem}

\begin{proof}
$ f $ は単調増加関数であるから,任意の $ x, x' \in I $ について以下が成り立つ.
\begin{align*}
& x < x' \Rightarrow f(x) \leq f(x') \\
\Longleftrightarrow \hspace{8pt} & x < x' \Rightarrow -f(x) \geq -f(x')
\end{align*}
これは $ -f $ が単調減少関数であることを示し,$ f $ が単調増加関数であることと,$ -f $ が単調減少関数であることが同値であることが証明された.
\end{proof}

定理 \ref{keeping_monotonicity} より,関数値を $ -1 $ 倍する操作によって単調増加と単調減少が切り替わり,単調性は失われないことがわかる.さらに,最適化問題において,ある関数 $ f $ の最大化は $ -f $ の最小化と同じであるため,関数値を $ -1 $ 倍することで最大化問題を最小化問題に変換することができる.ここからは単調増加関数の最小化問題についてしか論じないが,この通り最大化問題は最小化問題に変換することができ,単調性は保存されることから,一般性は失わない.

\section{単体}

\begin{definition}
以下を単位単体という.
$$ \Delta_n \triangleq \Set{ x \in \mathbb{R}^n_{+} | \sum_{i=1}^{n} x_i = 1 } $$
\end{definition}

$ n $ について具体的に考えてみると,$ n=2 $ のとき線分,$ n=3 $ のとき平面上の正三角形,$ n=4 $ のとき三角錐となることがわかる.このように $n$ に対して一次元減った図形になることから,先ほど定義した $n$ 次元の単位単体 $ \Delta_n $ は $ n-1 $ 単体と呼ばれる.

\section{単体上の単調関数}

単調増加関数 $ f : \mathbb{R}^n_{+} \to \mathbb{R} $ の単位単体上における性質について考える.まず,単体の定義から以下が成り立つ.\\

\begin{theorem}
単体上の任意の点 $ x, y \in \Delta_n $ について $ x < y $ が成り立つことはない.
\end{theorem}

\begin{proof}
背理法で示す.$ x < y $ となるような $ x, y $ が $ \Delta_n $ 中に存在すると仮定する.\\
$ x < y $ であるから,
\begin{align}
& x_i < y_i \;, i = 1, ..., n \notag \\
\Longrightarrow \hspace{8pt} & \sum_{i=1}^n x_i < \sum_{i=1}^n y_i \notag
\end{align}
これは $ x, y \in \Delta_n $ つまり,$ \sum_{i=1}^n x_i = \sum_{i=1}^n y_i = 1 $ に矛盾する.\\
よって,$ x < y $ が成り立つような単体上の点 $ x, y $ は存在しないことが証明された.
\end{proof}

つまり,単調増加関数の唯一の性質である $ x < y \Rightarrow f(x) < f(y) $ を利用できる点が単体上には存在せず,単体上だけを見ると特に性質のない任意の関数であるように見える.以上のことから,単体上における単調関数の最適化は一般的な大域的最適化問題を解くことと同程度に困難であると予想される.\par
しかし,実行可能領域以外の領域の情報を利用し,単体上における下界を得ることができる.\\

\begin{theorem}\label{lower_bound_of_monotonic_function}
以下に示す $ x_{LB} \in \mathbb{R}^n_{+} $ は単体の任意の部分領域 $ A \subseteq \Delta_n $ における単調増加関数 $ f :  \mathbb{R}^n_{+} \to \mathbb{R} $ の関数値の下界 $ f(x_{LB}) $ を与える.
$$ x_{LB} = \Set{ x \in \mathbb{R}^n_{+} | x_i = \min_{y \in A } y_i, \; i = 1, ..., n } $$
\end{theorem}

\begin{proof}
$ x_{LB} $ の生成方法より,$ A $ の任意の要素 $ x $ について $ x_{LB} \leq x $ が成り立つ.\\
また,$ f $ は単調増加関数であったから,$ \forall x, y \in \mathbb{R}^n_{+}, x < y \Rightarrow f(x) < f(y) $ が成り立ち,
$$ \forall x \in A, f(x_{LB}) \leq f(x) $$
よって,$ f(x_{LB}) $ は単体の任意の部分領域 $ A $ における単調増加関数 $ f $ の下界となる.
\end{proof}

\chapter{関連研究}

\section{分枝限定法}

分枝限定法とは,大域的最適化の汎用アルゴリズムであり,広く利用されている.実行可能領域を分割し,それぞれの分割された領域に対して下界値・上界値を求め,それらや暫定最適値を比較することで最適解が存在し得ない領域を削除し,探索する領域を絞り込む.以上を繰り返して効率的に探索をする最適化手法である.\par
最小化問題を解く場合の一般的なアルゴリズムとして分枝限定法の説明を行う.分枝限定法は大きく分けて分枝・限定の2種類の操作から成る.分枝操作では,実行可能領域から領域 $ S $ を選択し,$ S_1 \cup S_2 \cup ... \cup S_n = S, S_1 \cap S_2 \cap ... \cap S_n = \phi $ を満たす $ n $ 個の部分領域に分割する.分割の方法や分割数はアルゴリズムによって異なり,問題に対して適した手法を考えることが重要である.次に限定操作では,それぞれの部分領域について下界値を求める.その下界値が,他の領域の上界値よりも大きい場合や,暫定最適値よりも大きい場合はその部分領域に最適解が存在し得ないため,その領域の探索を打ち切る.ある領域の探索を打ち切り,削除することを刈り込みと呼ぶ.領域の刈り込みを効率的に行うには,よい下界,上界を得ること,探索中に素早くよい暫定最適値を見つけることが重要である.以上の操作をすべての領域を刈り込むまで繰り返し実行し,その時点の最適値を問題の最適値として出力する.\par

\section{単体のグリッド化}

de Klerk \cite{deklerk} などにより,単体をグリッド化して探索を行う研究がされている.ここでグリッド化とは,単体の内部に格子を考え,格子の交差点だけを抽出し,単体の内部に等間隔に存在する点群を得ることである.グリッド化によって実行可能領域は小さくなり,解きやすくなる.まずは単体のグリッド化の定義を行う.\par
$ m \in \mathbb{Z} $ を利用して,$ n $ 次元の単位単体の各辺を $ m $ 分割して得られるグリッド化された単位単体は以下のように表せる.
$$ \Set{ x \in \Delta_n | mx \in \mathbb{Z}^n } $$
これまでは単位単体のみを扱ってきたが,正の実数 $ c $ を利用して,各辺の長さが $ c $ であるような単体 $ c \Delta_n $ をグリッド化したもの $ G(c, n, m) $ を以下のように定義する.\par
\begin{definition}
$$ G(c, n, m) \triangleq c \Delta_n \cap \Set{x \in \mathbb{R}^n | mx \in \mathbb{Z}^n} $$
\end{definition}
ここで,$ c = 1 $ としたとき,$ G(1, n, m) $ は $ n $ 次元の単位単体の各辺を $ m $ 等分してグリッド化したものと同じであることがわかる.\par
以下は $ n = 3, m = 5 $ のときのグリッド化された単体の図である.\par
% TODO: 図を描く

\begin{theorem} \label{theorem:n_grid}
グリッド化された単位単体 $ G(1, n, m) $ が持つ点の数は二項係数を利用した以下で与えられることがわかっている.
$$ | G(1, n, m) | = \binom{n + m - 1}{m} $$
\end{theorem}
これは $ m $ を固定した場合の $ n $ の多項式であり,最適化手法としてグリッド化した単体上を全探索しても $ n $ の多項式オーダーでアルゴリズムが完了することがわかる.しかし,もちろん $ n, m $ を大きくした場合の $ \binom{n + m - 1}{m} $ はとても大きい数であり,例えば $ n = 10, m = 100 $ の場合,生成される点の数はおよそ $ 8.5 \times 10^{12} $ であり,全探索は難しいと容易に想像できる.

\section{単体の分割} \label{partition}

分枝限定法の分枝操作では実行可能領域をいくつかの部分領域に分割するが,本問題の実行可能領域は単体であるから,単体を分割する手法を考える.さらに,分枝操作は繰り返し行われるため,分割された部分領域も同様にすべて単体となっている必要がある.\par
外崎ら \cite{tonosaki} による単体の分割手法は,グリッド化された単体を一次元少ない単体と,一列少ない単体の二つに分割する方法である.この手法を具体的に説明する.\par
$ mb \in \mathbb{Z}^n $ を満たすような非負ベクトル $ b \in \mathbb{R}^n_{+} $ を利用して $ c $ を次のように与える.
$$ c \equiv 1 - \sum_{j=1}^{n} b_j > 0 $$
このとき次元のインデックスの部分集合 $ K = \{j_1, ..., j_k\} \subset \{1, ..., n\} $ を選択すると,単体上の最小化問題である元の問題 \ref{math:primal_problem} の部分問題を以下のように定義することができる.

$$
P(K, b) \;
\left| \;
\begin{aligned}
& 最小化 && f(x) \\
& 条件 && \sum_{j = 1}^n x_j = 1 \\
& && mb_j \leq mx_j \in \mathbb{Z}, && j \in K \\
& && x_j = b_j, && j \not \in K
\end{aligned}
\right.
$$

$ K $ に選択されなかった次元は $ b $ の値に固定され,$ K $ に選択された次元の変数のみを変数とするような問題となる.この問題は以下の問題と等価である.

$$
\left| \;
\begin{aligned}
& 最小化 && f(x) \\
& 条件 && y \in G(c, k, m) \notag \\
& && y_i = x_{j_i} - b_{j_i}, && j = 1, ..., k \notag \\
& && x_j = b_j, && j \not \in K \notag
\end{aligned}
\right.
$$

この問題は各辺の大きさが $ c $ である $ k $ 次元のグリッド化された単体上の問題である.さらに定義より,$ G(c, k, m) $ は各辺を $ mc $ 個に分割したグリッドの集合であり,定理 \ref{theorem:n_grid} より以下が成り立つ.
$$ |G(c, k, m)| = \binom{k + mc -1}{mc} $$
次に,$ K $ から一つのインデックス $ j_k $ を選択し $ P(K, b) $ を二つの部分問題に分割する.

$$
P(K, b + e_{j_k} / m) \;
\left| \;
\begin{aligned}
& 最小化 && f(x) \\
& 条件 && \sum_{j = 1}^n x_j = 1 \\
& && mb_j \leq mx_j \in \mathbb{Z}, && j \in K \backslash \{ j_k \} \\
& && mb_{j_k} + 1 \leq mx_{j_k} \in \mathbb{Z} \\
& && x_j = b_j, && j \not \in K
\end{aligned}
\right.
$$

$$
P(K \backslash \{ j_k \}, b) \;
\left| \;
\begin{aligned}
& 最小化 && f(x) \\
& 条件 && \sum_{j = 1}^n x_j = 1 \\
& && mb_j \leq mx_j \in \mathbb{Z}, && j \in K \backslash \{ j_k \} \\
& && x_j = b_j, && j \not \in K \\
& && x_{j_k} = b_{j_k}
\end{aligned}
\right.
$$

これらはそれぞれ以下の2問題と同値である.

$$
\left| \;
\begin{aligned}
& 最小化 && f(x) \\
& 条件 && y \in G(c - 1/m, k, m) \notag \\
& && y_i = x_{j_i} - b_{j_i}, && i = 1, ..., k-1 \notag \\
& && y_k = x_{j_k} - b_{j_k} - 1 / m \notag \\
& && x_j = b_j, && j \not \in K \notag
\end{aligned}
\right.
$$

$$
\left| \;
\begin{aligned}
& 最小化 && f(x) \\
& 条件 && y \in G(c, k - 1, m) \notag \\
& && y_i = x_{j_i} - b_{j_i}, && i = 1, ..., k-1 \notag \\
& && x_j = b_j, && j \not \in K \notag \\
& && x_{j_k} = b_{j_k} \notag
\end{aligned}
\right.
$$

$ P(K, b + e_{j_k} / m) $ は $ P(K, b) $ からグリッドが一列減った問題であり,$ P(K \backslash \{ j_k \}, b) $ は次元が1次元減った問題である.以上のようにグリッド化された単体を,一列減った単体と一次元減った単体に分割することができた.もともと単体には一列という概念は存在しないが,グリッド化をしたことによってこのような操作が可能になった.さらに,これらの定義域のグリッド化された単体のグリッド数について見てみると,以下がわかる.
$$ | G(c - 1/m, k, m) | = \binom{k + mc - 2}{mc - 1} $$
$$ | G(c, k - 1, m) | = \binom{k + mc - 2}{mc} $$
二項係数の性質として以下が知られている.
$$ \binom{k + mc - 1}{mc} = \binom{k + mc - 2}{mc - 1} + \binom{k + mc - 2}{mc} $$
ゆえに,以下が成り立つ.
$$ | G(c, k, m) | = | G(c - 1/m, k, m) | + | G(c, k - 1, m) | $$
この分枝操作を繰り返すと,すべてのグリッド化された単体はただ一点のみを含む単体になり,その総数は $ \binom{k + mc - 1}{mc} $ となる.\par
元の問題 \ref{math:primal_problem} から分枝操作を繰り返すことで,問題をノードに持つ二分木 $ T $ を考えることができ,その木の葉は一点のみを含む単体となる.この二分木のノード数について以下のことが言える.

\begin{theorem}
$ T $ の総ノード数は $ 2 \binom{n + m - 1}{m} - 1 $ である
\end{theorem}
\begin{proof}
帰納的に示す.$ r = \binom{n + m - 1}{m} $ とおく.$ q < r $ であるような $ q $ に対して,$ q $ を葉の総数とする二分木のノード数は $ 2q - 1 $ であると仮定する.\\
次に $ T $ の根を親に持つような二つの部分木を $ T_1, T_2 $ とし,それらの葉の数を $ r_1, r_2 $ とすると,$ r = r_1 + r_2 $ が成り立つ.さらに,$ r_1, r_2 \leq r - 1 $ であるから仮定より,$ T_1, T_2 $ のノード数は $ 2r_1 - 1, 2r_2 - 1 $ である.\\
よって,$ T $ のノード数は
$$ (2r_1 - 1) + (2r_2 - 1) + 1 = 2(r_1 + r_2) - 1 = 2r - 1 $$

\end{proof}

\section{単調関数の最適化}

Tuy による,単調関数の最適化.

\chapter{提案手法}

単体上の単調関数を分枝限定法で最適化する.分枝限定法において,領域の分割方法,下界の良さはアルゴリズムの速度に対して大変重要であり,本研究では単調関数の下界に定理  \ref{lower_bound_of_monotonic_function} で得られるものを利用し,分割方法は \ref{partition} で説明したもの利用する.

\section{アルゴリズム}



\chapter{数値実験}
\section{条件}
\section{実験結果}

\chapter{結論}

\chapter*{謝辞}
\addcontentsline{toc}{chapter}{\numberline{}謝辞}

ありがとうございました.

\newpage

\addcontentsline{toc}{chapter}{\numberline{}参考文献}
\renewcommand{\bibname}{参考文献}

\begin{thebibliography}{1}

% TODO: 引用の仕方を統一する
\bibitem{tuy}
H. Tuy,
\newblock Monotonic optimization: Problems and solution approaches,
\newblock SIAM Journal on Optimization 11.2 (2000), pp. 464-494.

\bibitem{deklerk}
de Klerk E,
\newblock The complexity of optimizing over a simplex, hypercube or sphere: a short survey,
\newblock Central European Journal of Operations Research 16.2 (2008), pp. 111-125.

\bibitem{tonosaki}
外崎真造, 久野誉人,
\newblock 非凸2次制約付き配合計画問題のロバスト最適化,
\newblock 筑波大学システム情報工学研究科修士学位論文 (20010)
\end{thebibliography}

\end{document}
